\documentclass{article}

% Paquetes comunes
\usepackage[utf8]{inputenc}
\usepackage[T1]{fontenc}
\usepackage[spanish]{babel}
\usepackage{amsmath}
\usepackage{graphicx}
\usepackage{booktabs}
\usepackage{url}
\usepackage{hyperref}

\title{Mi Primer Documento Co56546mpleto en \LaTeX}
\author{Nombre Apellbjnido}
\date{\today}

\begin{document}

\maketitle

\begin{abstract}
Este es un documento de ejemplo para probar una instalación de \LaTeX, que incluye texto, una imagen, una tabla y una referencia bibliográfica. El objetivo es verificar que la compilación se realiza correctamente y que todos los elementos se muestran en el PDF final.
\end{abstract}

\section{Intghjrocción}
Este documento ha sido creado para demostrar las capacidades básicas de \LaTeX, un sistema de composición de textos muy utilizado en el ámbito académico y científico.

\subsection{Texto de ejemplo}
Aquí tenemos algo de texto simple. Podemos usar negritas con \textbf{texto en negritas} y cursivas con \textit{texto en cursivas}. También podemos citar una referencia de la bibliografía, como se muestra en la referencia \cite{greenwade93}.

\section{Imágenes y figuras}
A continuación, se inserta una figura. Para que funcione, debes colocar un archivo de imagen llamado \texttt{logo.png} en la misma carpeta que este archivo. Si no tienes una, puedes descargar una pequeña imagen o simplemente omitir esta sección.



En la figura \ref{fig:logo} podemos ver una ilustración. La referencia `\ref{fig:logo}` genera el número de la figura.

\section{Tablas}
También es posible crear tablas profesionales. La tabla \ref{tab:ejemplo} muestra una tabla de ejemplo con datos de tres columnas.

\begin{table}[h]
    \centering
    \caption{Tabla de ejemplo.}
    \label{tab:ejemplo}
    \begin{tabular}{lcr}
        \toprule
        Encabezado 1 & Encabezado 2 & Encabezado 3 \\
        \midrule
        Fila 1 & 123 & 456 \\
        Fila 2 & 789 & 1011 \\
        Fila 3 & 1213 & 1415 \\
        \bottomrule
    \end{tabular}
\end{table}

\section{Bibliografía}
Para que la bibliografía funcione, necesitas crear un archivo `.bib` con el mismo nombre que el archivo `.tex` y usar la misma referencia.

\begin{thebibliography}{99}

\bibitem{greenwade93}
George D. Greenwade.
\newblock The Comprehensive \TeX\ Archive Network (CTAN).
\newblock {\em TUGboat}, 14(3):342--351, 1993.

\end{thebibliography}

\end{document}
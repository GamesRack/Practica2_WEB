\documentclass[12pt]{article}
\usepackage{graphicx} % Required for inserting images
\usepackage[spanish]{babel}
\usepackage{lmodern}
\renewcommand{\familydefault}{\sfdefault}
\usepackage{geometry}
\usepackage{setspace}
\usepackage{float}
\usepackage{xcolor}
\usepackage{amsmath}
\usepackage{tikz}
\usepackage[colorlinks=true, urlcolor=blue]{hyperref}
\usepackage{hyperref}
\title{Practica 2 - WEB}
\author{Cerón Samperio Lizeth Montserrat}
\date{September 2025}

\begin{document}
\definecolor{azul}{RGB}{0, 168, 255}
\definecolor{azul2}{RGB}{7, 74, 163}

\begin{titlepage}
    \thispagestyle{empty}
    \newgeometry{left=2cm, right=1cm, top=2cm, bottom=2cm}
    
    % Gráfico de fondo corregido para ser fiable
    \begin{tikzpicture}[overlay, remember picture, fill opacity=0.7]
        \begin{scope}[shift={(current page.center)}]
            \rotatebox{-45}{\fill[azul2] (5, -15) rectangle (12, 15);}
            \rotatebox{-45}{\fill[azul] (3, -15) rectangle (5, 15);}
        \end{scope}
    \end{tikzpicture}
    
    \begin{spacing}{1.5}
        {\Huge \bfseries \noindent PRÁCTICA 2}
        \vspace{10pt}

        {\LARGE MÉTODOS HTTP: PUT, PATCH, DELETE, GET}
        
        \vspace{1cm}
        
        {\Large Equipo:} \\
        {\Large Beltrán Saucedo Axel Alejandro} \\
        {\Large Cerón Samperio Lizeth Montserrat} \\
        {\Large Higuera Pineda Angel Abraham} \\
        {\Large Lorenzo Silva Abad Rey} \\
        {\Large 4BV1}
    \end{spacing}
    
    \vspace{1.5cm}

    \begin{minipage}{7.5cm} % Ancho ajustado
        {\Large ESCUELA SUPERIOR DE CÓMPUTO}
    \end{minipage}

    \vfill % Empuja el contenido hacia el final de la página

    \begin{flushleft}
        {\Large \color{black}
        % Comando \textbf corregido con llaves
        \textbf{TECNOLOGÍAS PARA EL DESARROLLO DE APLICACIONES WEB}}
        
        \vspace{0.5cm}
        
        25/09/2025
    \end{flushleft}
    \vspace{1cm}
\end{titlepage}


\section{Introducción}
\subsection*{Objetivo de la práctica:}
El objetivo de la práctica es implementar los métodos HTTP: PUT, PATCH, GET, DELETE en Python con FastAPI, también realizar pruebas de cada uno de los métodos, poniendo a prueba las respuestas que brindará. La aplicación simulará la gestión de paquetes, es decir, que se puede crear un item nuevo con peso y valor, también se le puede poner un ID para ejecutar las funciones de borrar y actualizar parcialmente por el ID. \\


\subsection*{Métodos HTTP:}
\begin{itemize}
    \item \textbf{GET:} Este método se utiliza para solicitar datos de un recurso específico. En la práctica, se implementa para obtener la lista completa de paquetes o un paquete específico por su ID.
    \item \textbf{POST:} Este método se utiliza para enviar datos a un servidor para crear un nuevo recurso. En la práctica, se implementa para crear un nuevo paquete con atributos como peso y valor.
    \item \textbf{PUT:} Este método se utiliza para actualizar completamente un recurso existente. En la práctica, se implementa para actualizar todos los atributos de un paquete específico identificado por su ID.
    \item \textbf{PATCH:} Este método se utiliza para actualizar parcialmente un recurso existente. En la práctica, se implementa para modificar solo algunos atributos de un paquete específico identificado por su ID.
    \item \textbf{DELETE:} Este método se utiliza para eliminar un recurso específico. En la práctica, se implementa para eliminar un paquete identificado por su ID. \cite{ref1}
\end{itemize}

\section{Fundamentos teoricos}
\subsection*{FastAPI:}
FastAPI es un framework web para construir APIs en Python, lanzado en diciembre de 2018 por Sebastián Ramírez, un desarrollador colombiano. Está diseñado para ser:
\begin{itemize}
    \item \textbf{Rápido:} comparable en rendimiento a Node.js y Go gracias a su base en ASGI (Asynchronous Server Gateway Interface).
    \item \textbf{Moderno:} aprovecha las anotaciones de tipo de Python 3.6  
    \item \textbf{Automático:} genera documentación interactiva con Swagger UI y ReDoc.
    \item \textbf{Seguro y robusto:} ideal para APIs que requieren validación estricta y autentic
\end{itemize}
Pero,  ¿qué hace realmente FastAPI?

Nos permite crear endpoints o rutas para nuestra API, que pueden ser accedidas mediante los diferentes metodos HTTP, como GET, POST, PUT, DELETE, entre otros.\\
Además nos permite:
\begin{itemize}
    \item Validar automaticamente los datos de entrada y salida utilizando Pydantic.
    \item Servir modelos de machine learning en producción. 
    \item Integrarse facilmente con base de datos, etc.
    \item Generar documentación interactiva sin esfuerzo.
\end{itemize}

Pero, ¿para qué lo usamos realmente?\\
Usamos FastAPI para crear aplicaciones que se comunican por internet, como por ejemplo:
\begin{itemize}
    \item Aplicaciones web que mandan y reciben datos (como una app de clima o una tienda en línea).
    \item Sistemas que conectan con modelos de inteligencia artificial (como chatbots o análisis de imágenes).
    \item Servicios que otras apps usan para pedir información (como “dame los datos del usuario” o “guárdame esta foto”).
\end{itemize}

\subsection*{HTTP:}
FastAPI utiliza HTTP para comunicarase entre la aplicación web y el cliente. Cada vez que alguien hace una petición (Por ejemplo: GET, POST, etc) a nuestra API, FastAPI recibe esa petición, la procesa y responde con los datos solicitados.\\

¿Qué es HTTP?\\
HTTP (Hypertext Transfer Protocol). Es un lenaguaje utilziado para realizar una comunicación entre un cliente (como un navegador web o una app móvil) y un servidor (donde vive la aplicación web).\\
Funciona con peticiones y respuestas, aglunas de ellas son:
\begin{itemize}
    \item \textbf{GET:} Solicita datos de un recurso específico.
    \item \textbf{POST:} Envía datos para crear un nuevo recurso.
    \item \textbf{PUT:} Actualiza completamente un recurso existente.
    \item \textbf{PATCH:} Actualiza parcialmente un recurso existente.
    \item \textbf{DELETE:} Elimina un recurso específico.
\end{itemize}

Tambien se cuentan con las HTTP Exceptions, que son respuestas a errores que ocurren cuando algo sale mal en la comunicación entre el cliente y el servidor.\\
Algunos ejemplos son:
\begin{itemize}
    \item \textbf{404:} No encontrado
    \item \textbf{403:} Prhohibido
    \item \textbf{500:} Error interno del servidor
    \item \textbf{400:} Solicitud incorrecta  
\end{itemize}

\section{Diagrama UML}
\begin{figure}[H]
    \centering
    \includegraphics[width=1\textwidth]{Imagenes/Diagrama UML.png}
\end{figure}
\section{Implementación}
\begin{figure}[h!]
    \centering
    \includegraphics[width=1\textwidth]{Imagenes/Captura1_librerias.png}
\end{figure}

En este apartado se tienen los modulos a utilziar para el funcionamiento de la practica.\\
FastApi nos permite contruir APIS de manera rapida y sencilla, permitiendo definir rutas y manejar solicitudes HTTP de manera eficiente.\\
Las rutas que utilizaremos en esta practica son: GET, POST, PUT y DELETE.\\
Utilizamos Pydantic para validad entradas y para la creacion de un modelo de datos, que en es este caso sera para tener la id de un objeto, y su respectivo contenido.\\
Optional y list son tipos de datos que nos permiten definir atributos que pueden ser opcionales y listas respectivamente.\\


\begin{figure}[h!]
    \centering
    \includegraphics[width=1\textwidth]{Imagenes/Captura2_corazon del programa.png}
\end{figure}

Iniciamos con el corazon de nuestro programa, que es la instancia de FastAPI, en donde definimos el nombre y la descripcion de la API.\\

\begin{figure}[H]
    \centering
    \includegraphics[width=1\textwidth]{Imagenes/Captura3_esctructuradatos.png}
\end{figure}

Gracias a Pydantic, podemos crear una estructura de datos para nuestros objetos a utilizar.
Se tiene un id de tipo entero, este servira para identificar cada objeto.
Cda objeto tiene un contenido de datos, en este caso cuenta con ganancia y perso, ambos de tipo flotante.
Además se tiene una estrcutura para actualizar los datos parcialmente, los campos son opcionales, esto es debido a que no es obligatorio actualizar ambos.

\begin{figure}[H]
    \centering
    \includegraphics[width=1\textwidth]{Imagenes/Captura4_variablesGlobales.png}
\end{figure}
Utilizamos variables globales para almacenar los objetos y llevar un control del id.
El id se inicializa en 0, y se incrementa cada vez que se crea un nuevo objeto.
Los objetos se almacenan en una lista, que inicialmente esta vacia.


\begin{thebibliography}{99}

\bibitem{ref1} M\'etodos de petici\'on HTTP. (s/f). \textit{MDN Web Docs}. Recuperado el 23 de septiembre de 2025, de \url{https://developer.mozilla.org/es/docs/Web/HTTP/Reference/Methods}

\bibitem{ref2} Ram\'irez Monta\~no, S. (s.f.). ``Historia, dise\~no y futuro de FastAPI''. \textit{FastAPI}. Recuperado el 23 de septiembre de 2025, de \url{https://fastapi.tiangolo.com/es/history-design-future/}

\bibitem{ref3} Wikipedia. (2025, julio 11). ``FastAPI''. \textit{Wikipedia, la enciclopedia libre}. Recuperado el 23 de septiembre de 2025, de \url{https://es.wikipedia.org/wiki/FastAPI}

\bibitem{ref4} Lubanovic, B. (2019). \textit{Introducing Python: Modern Computing in Simple Packages} (2\textsuperscript{a} ed.). O'Reilly Media. ISBN: 9781492051367

\bibitem{ref5} Linode Guides \& Tutorials. (2021, agosto 6). ``Document a FastAPI App with OpenAPI''. \textit{Linode}. Recuperado el 23 de septiembre de 2025, de \url{https://www.linode.com/docs/guides/document-a-fastapi-app-with-openapi/}

\end{thebibliography}
\end{document}